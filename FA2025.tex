% -----------------------------------------------
% Template for FA2025 Proceedings

% DO NOT MODIFY THE FOLLOWING SECTION!!
%-------------------------------------
\documentclass[11pt]{article}
\usepackage{fa2025}
\usepackage{amsmath}
\usepackage{cite}
\usepackage{url}
\usepackage{graphicx}
\usepackage{color}
\usepackage{siunitx}
\usepackage[utf8]{inputenc}
%-------------------------------------


% \title{Comparing without classifying: evaluating embeddings from bioacoustic deep learning feature extractors}
\title{Clustering and novel class recognition: evaluating embeddings from bioacoustic deep learning feature extractors}

\multauthor{
  Vincent S. Kather$^{1, 2*}$ 
  \hspace{1cm} 
  Burooj Ghani$^2$ 
  \hspace{1cm} 
  Dan Stowell$^{1, 2}$
  } { 
  $^1$ Department of Cognitive Science and Artificial Intelligence, Tilburg University, Netherlands\\
  $^2$ Naturalis Biodiversity Center, Leiden, Netherlands\\
% $^3$  Company, Address
  \correspondingauthor{vincent.kather@naturalis.nl}{Vincent S. Kather et al.}
}



\sloppy % please retain sloppy command for improved formatting
\begin{document}

%
\maketitle
\begin{abstract}
% In computational bioacoustics, deep learning models are composed of feature extractors and classifiers. 
% The feature extractors generate vector representations of the input sound segments, called embeddings. 
% The classifiers recognize a fixed number of classes most commonly representing different animal species. 
% Various benchmarks have been published to evaluate the classification performance. 
% While such benchmarking provides insight into specific performance statistics, it is limited to species that are included in the models training data. Furthermore, it makes it impossible to compare models trained on very different taxonomic groups. 
% This paper aims to address this gap by analyzing the generated embeddings of more than 15 bioacoustic models spanning a wide range of setups (model architectures, training data, training paradigms). 
% We investigate and evaluate different ways to quantify how models structure embedding spaces, which allows us to focus our comparison on feature extractors independent of classifiers. 
% We believe that this approach lets us evaluate the adaptability and generalization potential of models going beyond the classes they were trained on.
In computational bioacoustics, deep learning models are composed of feature extractors and classifiers. 
The feature extractors generate vector representations of the input sound segments, called embeddings. 
% The classifiers recognize a fixed number of classes most commonly representing different animal species. 
% Various benchmarks have been published to evaluate the classification performance. 
While benchmarking of classification scores provides insight into specific performance statistics, it is limited to species that are included in the models training data. Furthermore, it makes it impossible to compare models trained on very different taxonomic groups. 
This paper aims to address this gap by analyzing the embeddings generated by the feature extractors of 15 bioacoustic models spanning a wide range of setups (model architectures, training data, training paradigms). 
We evaluate and compare different ways how models structure embedding spaces through clustering and kNN classification, which allows us to focus our comparison on feature extractors independent of their classifiers. 
We believe that this approach lets us evaluate the adaptability and generalization potential of models going beyond the classes they were trained on.
\end{abstract}

\keywords{\textit{
  deep learning, bioacoustics, embeddings
  }}
%

\section{Introduction}
\label{sec:introcution}

Human-driven climate change and deforestation have caused a rapid decline in global biodiversity \cite{butchart_global_2010}.
Using sensor arrays for biodiversity monitoring, ecologists can gather information on environments and investigate how human pressures affect biodiversity and how we can halt its decline \cite{schmeller_building_2017}.
Passive acoustic monitoring (PAM) is one method that provides a low-cost and non-invasive way to monitor biodiversity \cite{sugai_terrestrial_2019}.
The vast amount of data generated by PAM sensors has led to the rapid development of bioacoustic deep learning models to help researchers reduce the annotation effort \cite{stowell_computational_2022}.
The use of these state-of-the-art models has proven valuable in ecological studies, for example in general species assessments \cite{tuia_perspectives_2022,cowans_improving_2024} or detection of endangered species \cite{allen-ankins_use_2025}.

While bioacoustic deep learning models are a useful asset in ecology, it is crucial to understand the model's limitations based on their training setup.
% little is known about how their training setup correlates to their performance.
The two main training strategies to develop bioacoustic deep learning models are supervised learning and self-supervised learning.
% In supervised learning large acoustic datasets are collected, annotated and used to train \cite{kahl_birdnet_2021}. 
Supervised learning models require large amounts of annotated data to be trained \cite{hagiwara_aves_2022}.
The models classify sounds based on a fixed number of predefined classes representing annotated sounds in the dataset.
While supervised learning can have the benefit of instructing models to differentiate between species vocalizations, it requires annotated datasets.
This limits supervised learning models to known and annotated classes and makes them sensitive to class imbalance and label quality.
% In self-supervised learning classes are defined by the model during training time \cite{baevski_efficient_2023,huang_masked_2022}.
Self-supervised learning can be executed in different ways.
In a paradigm referred to as masked prediction, the model is trained to predict a masked portion of the audio, thereby modelling bioacoustic characteristics without supervision \cite{baevski_efficient_2023,huang_masked_2022}.
With growing annotated databases supervised learning models like Birdnet \cite{kahl_birdnet_2021} improve in performance, yet recent developments in self-supervised learning on models like animal2vec \cite{schafer-zimmermann_animal2vec_2024} indicate a promising new direction for the field requiring less manual annotation.
Not requiring annotations, self-supervised learning models can be trained on far larger datasets, however there is no control of what is being learned.
The model might learn to differentiate between sounds based on very different characteristics than the species that produces them.

% The datasets that are created in this manner tend to feature many annotations for few classes and few annotations for many classes (long-tail distribution/class imbalance) \cite{arnaud_improving_2023}. 
% Models with a heavily imbalanced training dataset have shown to yield performance imbalances \cite{hamer_birb_2023}. 
% This implies that the models results are only reproducible when used to classify the well represented majority classes. 
% Furthermore, models trained in a supervised learning pipeline are restricted to only recognize a fixed set of classes. 
% Class imbalance and closed set recognition are two examples of phenomena, that show while supervised learning is the convention, it has shortcomings that restrict the usability of models trained in this way.

To evaluate bioacoustic deep learning models, a comparison of only the classifier performance obscures the fine-grained differences between models and how they analyze input sounds.
Bioacoustic deep learning models consist of artificial neural networks which can be subdivided into feature extractors and classifiers. 
The feature extractor creates an embedding (vector representation) of an input sound and the classifier (which corresponds to the final dense layer of the model) maps the embedding onto classes.
Commonly, in bioacoustics, a suite of established benchmarks is used to compare the classifier performance of state-of-the-art models \cite{hamer_birb_2023}.
This requires the models to have been trained on the classes present in the benchmarking datasets.
However, there is an alternative: using the embeddings created by the feature extractors, the generated embedding spaces can be analyzed in regard to their structural characteristics, irrespective of what the classifiers were trained on.

Output dimensions of feature extractors vary greatly, but it is uncertain if their dimensionality correlates to the downstream classifier performance. 
Dimensionality reduction algorithms are useful to standardize the dimensionality of different feature extractors, as well as help visualize the high dimensional embedding spaces.
% Due to the lack of separable clusters in ecoacoustics and soundscape analyzes, evaluation of embedding spaces (reduced to two dimensions) has become common in recent years \cite{sethi_characterizing_2020,calonge_revised_2024,parcerisas_categorizing_2023}. 
However, there is little investigation of how reducing the embedding space affects performance.
We therefore compare performance in both original and reduced embedding spaces.

Due to their size, datasets that are used to train bioacoustic models for bird detection are often based on citizen science data (e.g. xeno-canto \cite{xeno-canto_xeno-canto_2025}).
The majority of these recordings are focal recordings of individuals which are weakly labeled with little polyphony.
When models get applied to large PAM datasets, these feature very different recording conditions.
To accurately evaluate the models in this study, we use a PAM bird vocalization dataset from Colombia and Costa Rica \cite{vega-hidalgo_collection_2023}, as well as a dataset of frog vocalizations in tropical rainforests of Brazil \cite{canas_dataset_2023}.
Both datasets are comprised of PAM recordings in noisy environments and can therefore be considered as challenging datasets.
Through the selection of challenging datasets, we hope to on the one hand emphasize the performance differences between the models and on the other hand produce results that will reflect in real world applications.

This study aims to showcase the potential for evaluating and especially comparing bioacoustic deep learning models in regard to their training paradigm and training data.
This evaluation is based on the structuring capabilities of their respective feature extractors analyzed through clustering and classification performance.
Classification in this case refers to recognition of novel classes, as none of the models have been pretrained on the classes selected here.
This way single classification layers are attached to each feature extractor, all of which are trained on the same evaluation sets (bird and frog vocalizations), i.e. same classes and same data, allowing us to compare their performance.
Classification is done both in a linear and a k-nearest neighbor (kNN) approach.
We perform our analysis in both the original embedding space and a reduced dimensional embedding space. 
That way we ensure the dimension is standardized for the second evaluation, whilst we can investigate performance differences between the original embedding space and the reduced space.
% This study will therefore examine how different dimensionality reduction strategies impact performance. 
% While the possibilities of analysis of these high dimensional embedding spaces are vast, we limit ourselves to 
% Performance will be evaluated through an analysis of clustering based on
% \begin{enumerate}
%     \item Silhouette Score \cite{rousseeuw_silhouettes_1987}, 
%     \item Adjusted Rand Index \cite{steinley_variance_2016} based on KMeans clustering and
%     \item Adjusted Mutual Information \cite{romano_standardized_2014} based on KMeans clustering.
% \end{enumerate}
% To do so, we analyse clusterings of deep learning feature extractors and use them to compare different training setups.
This method of analysis opens up the possibilities for a fair comparison of deep learning feature extractors guiding the field to a better understanding how training configurations affect downstream performance.
% Summary of what we are contributing.






\section{Methods}
\label{sec:methods}

To incorporate a variety of training setups, covering popular models as well as models targeted to various species groups, we compare a total of 15 pre-trained bioacoustic different feature extractors.
Table \ref{tab:bacpipe_models} shows the different feature extractors along with their model specific training setup. 
As can be seen both self-supervised and supervised learning feature extractors are represented. 
Furthermore, large variations in input length, embedding dimension and training data provide a landscape of feature extractors, allowing us to analyze performance of differently structured embedding spaces.

\subsection{Dataset}
\label{ssub:dataset}

The evaluation dataset that was used for this study is a collection of soundscape recordings from neotropical coffee farms in Colombia and Costa Rica \cite{vega-hidalgo_collection_2023}.
The recordings in this dataset feature challenging soundscape recordings with overlapping vocalizers and noisy environments.
The annotations are made for bird species.
The dataset has been reduced from its original size to only include sound events corresponding to classes with more than 150 annotations.
This results in 11 vocalizing bird species ranging in annotation count from 153 to over 4000 (see legend in Fig. \ref{fig:embeds}).
We intentionally selected soundscape recordings with gradual changes of background noise and overlapping species vocalizations to amplify the differences between the feature extractors' capabilities to structure the data.



\begin{table*}[t]
  
  \caption{List of feature extractors compared in this study. Columns "abbrev." shows the an abbreviated name used in Fig. \ref{fig:orig_vs_ump}. "training" shows the training setup chosen during training, i.e. ssl for self-supervised learning, supl for supervised learning and ft for fine-tuning. The "architecture" column more specifically describes the model architecture used. "dimension" shows the output dimension of the feature extractor. "trained on" summarizes the training data of the model. "ref." provides the respective publication.}
  \label{tab:bacpipe_models}

  \centering
  \begin{tabular}{l|c|c|c|c|l|c}
     \hline
    name& abbrev. & 
    training & architecture &
    dimension & trained on & ref.\\
      %  ref paper&
      %  ref code&
      %  sampling rate&
    \hline
    Animal2vec\_XC      & a2v\_xc   & ssl & d2v2.0 & 768 & birds & \cite{schafer-zimmermann_animal2vec_2024}\\
    Animal2vec\_MK      & a2v\_mk  & ssl + ft & d2v2.0 & 1024& meerkats & \cite{schafer-zimmermann_animal2vec_2024}\\
    AudioMAE            & aud\_mae  & ssl + ft & ViT 	 & 768 & general & \cite{huang_masked_2022}\\
    AVES\_ESpecies      & aves   & ssl + ft & HuBERT 	 & 768 & general + animals & \cite{hagiwara_aves_2022}\\
    AvesEcho\_PaSST     & aecho   & supl & PaSST 	 & 768 & birds & \cite{ghani_generalization_2024}\\
    BioLingual          & bioling  & supl & CLAP 	 & 512 & animals + birds & \cite{robinson_transferable_2023}\\
    BirdAVES\_ESpecies  & birdaves   & ssl + ft & HuBERT 	 & 1024& general + birds & \cite{hagiwara_aves_2022}\\
    BirdNET             & brdnet   & supl & EffNetB0 	 & 1024& birds & \cite{kahl_birdnet_2021}\\
    Google\_Whale       &  g\_whale  & supl & EffNetB0 	 & 1280& whales & - \\
    Insect459NET        & i459 & supl & EffNetv2s 	 & 1280& insects & - \\
    Insect66NET         & i66 & supl & EffNetv2s 	 & 1280& insects & - \\
    NonBioAVES\_ESpecies& nonbioaves    & ssl + ft & HuBERT 	 & 1024& general + non-bio & \cite{hagiwara_aves_2022}\\
    Perch\_Bird         & perch    & supl & EffNetB0 	 & 1280& birds & - \\
    ProtoCLR            & p\_clr   & supl 	 & CvT-13 & 384 & birds & \cite{moummad_domain-invariant_2024}\\
    SurfPerch           &  s\_perch  & supl & EffNetB0 	 & 1280& coral reefs + birds & \cite{williams_leveraging_2024}\\
    % \hline
  \end{tabular}
\end{table*}



\subsection{Data pipeline}
\label{ssub:data_pipe}

For each of the feature extractors, the respective model code base was cloned, and the model was stripped of its classifier.
For both animal2vec feature extractors, outputs from the attention heads and input lengths are averaged, resulting in one embedding per input segment (as is the case with all other feature extractors).
Data is imported from the sound files, resampled to the model specific sample rate and padded to fit the model specific input length.
All the necessary code to reproduce the computations can be found in the repository \textbf{bacpipe}\footnote{\url{github.com/bioacoustic-ai/bacpipe}} (\textbf{b}io\textbf{a}coustic \textbf{c}ollection \textbf{pipe}line).

\subsection{Evaluating dimensionality reduction}
\label{ssub:eval_dim_reduc}

Our primary focus is the comparison of the two paradigms: supervised learning and self-supervised learning.
Furthermore, we are looking into how the data chosen for training affects the clustering capabilities of different feature extractors.


% Given the high dimensional nature of embedding spaces, any quantitative evaluation may be subject to the curse of dimensionality \cite{bellman_dynamic_1957}.
% This refers to the issues that arise when dealing with comparatively small number of datapoints in very high dimensional space.
% In our setup this is relevant as the dimensions vary greatly and any resulting distance based calculations will be affected differently if they are calculated spaces of varying dimensions. 
% Therefore, we aim to evaluate in what way reducing the dimension of embedding spaces impacts performance for each of the feature extractors.
% The underlying assumption is that by homogenizing the dimension of the very differently structured embedding spaces, we lay the groundwork for quantitatively comparing the different feature extractors.

% % Therefore, we must first assess how reducing the dimension affects the performance of each feature extractor.
% Initially, feature extractor performance will be analysed using clustering performance and subsequently the performance of the reduced embedding spaces will be compared in the same way and related back to the initial performance values.
% We will compare two versions of linear dimensionality reduction algorithms, principal component analysis (PCA) \cite{wold_principal_1987} and sparse principal component analysis (sPCA) \cite{zou_sparse_2006} as well as one non-linear dimensionality reduction algorithm, uniform manifold approximation projection (UMAP) \cite{mcinnes_umap_2020}.
% Clustering performance will be analysed using Silhouette Score (SS) \cite{rousseeuw_silhouettes_1987}, Adjusted Rand Index (AIR) \cite{steinley_variance_2016} and Adjusted Mutual Information (AMI) \cite{romano_standardized_2014}.

% Reducing dimensions using linear transformations has the advantage of preserving relative distances between all data points.
% This means that quantitative analyses involving distance based calculations can be performed in the lower dimensional space.
% Non-linear dimensionality reduction like UMAP however, creates a learned transformation based on a graph structure of the data and therefore disregards distances between data points \cite{mcinnes_umap_2020}.
% While it is claimed that UMAP preserves relative distances for within cluster points, transformed relative distances between clusters are not representative of distances in the original high dimensional space.

% To evaluate the changing embedding spaces as a function of the dimensionality reduction method, clustering performance will be evaluated. Adjusted Rand Index (AIR) \cite{steinley_variance_2016} and  
The clustering is computed using KMeans with the same number of clusters as classes in the ground truth. 
Clustering performance will be evaluated using Adjusted Mutual Information (AMI) \cite{romano_standardized_2014} to compare the KMeans clustering with the ground truth.
Adjusted Rand Index is not included in this study, as it focuses on how well data points are grouped in a clustering, whereas we are primarily interested how well the KMeans clustering agrees with the ground truth labels.
% Non-linear dimensionality reduction algorithms like UMAP create a learned transformation based on a graph structure of the data and therefore disregards distances between data points \cite{mcinnes_umap_2020}.
Silhouette Score is also not included in this comparison as the challenging dataset yielded very low performance and variance, making a meaningful comparison impossible.
% AMI and ARI require a clustering to be computed, to then quantify the agreement between that clustering and the ground truth and are therefore applicable to non-linear dimensionality reductions.
% AMI measures how well clusters share information while ARI quantifies how well points are grouped.

Clustering performance is evaluated in both the original embedding spaces and an embedding space reduced to 300 dimensions.
This way the embedding dimension is standardized and performance can be compared while controlling for this factor.
It also allows us to compare the performance of each model in their high dimensional original embedding space, as well as in a reduced dimension.
To preserve relative distances between data points, Principal Component Analysis (PCA), a linear dimensionality reduction is selected.
To visualize the embeddings in two dimensions, a non-linear dimensionality reduction algorithm, uniform manifold approximation projection (UMAP) \cite{mcinnes_umap_2020} is selected.

Performance is also evaluated by training a linear classifier on each of the embedding spaces.
Data is split into train, validation and test set in the ratio 0.65:0.15:0.2.
The classifier is trained on the 11 classes for 10 epochs with a batch size of 64 and a learning rate of 0.001. 
Performance is evaluated using a balanced macro accuracy score \cite{brodersen_balanced_2010} to handle the imbalance in class size.



\section{Results}
\label{sec:results}

\begin{figure*}[ht]
    \centerline{\framebox{
    \includegraphics[width=16.3cm]{Sections/imgs/normal_overview.png}}}
    \caption{Two-dimensional embedding spaces of all feature extractors, sorted descending by their clustering performance of AMI values (indicated next to their name) from top left to bottom right.
    Given the different input lengths of the feature extractors, the number of embeddings vary significantly.
    Colors correspond to the class labels, which are 11 different tropical bird species.}
    \label{fig:embeds}
\end{figure*}

% Embeddings spaces of are visualized using UMAP in are generated from the input data using all feature extractors and then the dimensionality reduction algorithm UMAP is used to visualize the data in two dimensions.
% The results are shown in Fig. \ref{fig:embeds}.

Two dimensional UMAP embeddings are shown in Fig. \ref{fig:embeds}.
The worst performing feature extractors, produce large unstructured clouds of mixed color, indicating that no significant clustering is achieved.
In the first and second row, feature extractors can be seen to separate the embeddings into meaningful clusters.
It is noticable that some feature extractors such as AvesEcho\_PaSST and ProtoCLR seem to generate more subclusters than most other feature extractors.
The seven best performing feature extractors are all trained using supervised learning and the top three additionally trained on bird vocalizations.
All three of the AVES models (BirdAVES, AVES and NonBioAVES) reach similar performances in spite of big differences in their fine-tuning datasets \cite{hagiwara_aves_2022}.

\begin{figure}[ht]
    \centerline{\framebox{
        \includegraphics[width=7.8cm]{Sections/imgs/scatterplot_clust_vs_class_nt_knn_normal.png}}}
        \caption{Comparison of feature extractors by learning paradigm and training data. Abbreviated names correspond to abbrev. column in Tab. \ref{tab:bacpipe_models}. Differences in color correspond to training paradigm and differences in symbols correspond to training data. The x-axis shows clustering results of AMI while the y-axis shows macro accuracy results of linear classification. Colors correspond to supervised learning and self-supervised learning feature extractors, while symbols separate bird and non-bird training data.}
        \label{fig:subl_vs_ssl}
    \end{figure}
    
To investigate how training setup and training data affect performance, Fig. \ref{fig:subl_vs_ssl} shows a scatterplot of the different feature extractors.
Performance is evaluated by macro accuracy of linear classification on the x-axis and AMI of clustering on the y-axis.
% AMI is chosen to evaluate clustering performance as we are primarily interested to see how well the KMeans clustering agrees with the ground truth.

When focussing on the y-axis, all self-supervised learning feature extractors (in red) reach clustering performances under 0.31.
Performance by linear classification is more equally distributed, however, again supervised learning feature extractors reach the three highest values.
Furthermore, Animal2vec\_XC, the only self-supervised learning feature extractor that was not fine-tuned, performs poorly by both clustering and linear classification.
Google\_Whale represents the only supervised learning feature extractor performing very poorly by clustering.

Comparing by training data, feature extractors trained on only or including bird datasets outperform the other feature extractors by linear classification and even more so by clustering.
When looking at the combined performance by clustering and linear classification, Animal2vec\_XC and ProtoCLR are the only two feature extractors trained on birds that perform poorly.
Biolingual, which was trained on large bird databases using a multi-modal approach performs well by clustering, but poorly by linear classification.

When referring back to Table \ref{tab:bacpipe_models} embedding dimension does not correlate with clustering or linear classification performance.
Furthermore, the only two feature extractors trained on marine sounds, Google\_Whale and SurfPerch (trained on birds and marine sounds) reach very different performances.
    
\begin{figure}[ht]
    \centerline{\framebox{
    \includegraphics[width=7.8cm]{Sections/imgs/scatterplot_clust_vs_class_neotrop_anuran_normal.png}}}
    \caption{Comparison of feature extractor performance in original high dimensional embedding space and reduced 300 dimensional space using PCA. 
    The x-axis shows clustering results of AMI while the y-axis shows macro accuracy results of linear classification. 
    Symbols denote supervised or self-supervised learning, while red and green correspond to bird and non-bird training data. 
    The grey line and black markers indicate performance in the reduced dimensional embedding space.}
    \label{fig:orig_vs_ump}
\end{figure}

To account for the differences in embedding dimension, we visualized the change in performance between the original embedding space and a standardized embedding dimension of 300, to which all embedding spaces were reduced to using PCA.
The results are shown in Fig. \ref{fig:orig_vs_ump}.
While the poorly performing models ProtoCLR and Animal2vec\_MK are able to increase their linear classification performance, Insect459, Insect66 and Google\_Whale slightly improve linear classification and clustering performance.
While linear classification performance remains similar, SurfPerch, BirdNET and AvesEcho\_PaSST all decrease in clustering performance.
For the remaining feature extractors, standardizing the dimension, does not affect performance significantly.

\section{Discussion}
\label{sec:discussion}

Although the self-supervised feature extractors represented in this study are trained on very large datasets, their clustering performance is inferior to most of the supervised learning feature extractors.
Nonetheless, their lack of supervision demonstrates generalization as the majority of self-supervised feature extractors improve performance on the frog data, both for clustering and classification.
Yet, as has been shown in previous studies \cite{ghani_global_2023}, supervised learning and training data consisting of large bird song databases yield the best results, even for challenging PAM datasets.
Perch and BirdNET \cite{kahl_birdnet_2021}, both of which are trained on thousands of classes of bird vocalizations, vastly outperform the other feature extractors by clustering and for the bird data also by classification. 
%It is worth mentioning that they do so, while being trained on standard EfficienNETs, rather than custom architectures developed for a specific task.
Despite the authors claims, the multi-modal feature extractor Biolingual, which performs very well by clustering but comparably poor by classification is outperformed by BirdNET and Perch.
SurfPerch \cite{williams_leveraging_2024} and AvesEcho\_PaSST \cite{ghani_generalization_2024}, both of which have largely benefited from Perch or BirdNET pretraining, perform well for both clustering and classification tasks.
Most notably AvesEcho\_PaSST is the only supervised learning feature extractor that demonstrates generalization by improving performance on the frog data.
This could be attributed to the models' innovative transformer architecture using patchout, thereby selectively dropping patches of spectrograms during training, whilst having learned from the well trained BirdNET embeddings through knowledge distillation \cite{ghani_generalization_2024}.


% Interestingly, Insetct66 and Insect459 both of which were trained on only insect sounds and far fewer classes than the bird-trained biolingual, vastly outperform it in classification.
As stated in the introduction, self-supervised learning models lack supervision and might therefore learn classes not meaningful to differentiate between species vocalizations.
When comparing Figures \ref{fig:embeds} and \ref{fig:orig_vs_ump} we observe poor clustering both in terms of AMI performance and by qualitative visual analysis of the embedding separation, which could be resulting from non-meaningful classes.
However, for the three AVES feature extractors, the kNN classifier is nonetheless able to learn a meaningful differentiation between the classes.
This could be attributed to the fact, that while they are self-supervised, the data used to train these models consisted of curated and non-sparse sound events, thereby increasing the likelihood that meaningful classes are learned.
Furthermore, the HuBERT architecture used for the AVES models uses tokenization through acoustic unit discovery \cite{hagiwara_aves_2022}.
These acoustic units are generated during training based on 39-dimensional Mel-frequency ceptral coefficients (MFCC) features and labeled using a K-Means clustering.
The transformer encoder then predicts these cluster labels.
This differs fundamentally from the training procedure of the Animal2vec models and AudioMAE, both of which rely on masked prediction.
In terms of classification, the results suggest that the MFCC features lead to more meaningful classes than the prediction of masked emebddings.
This is supported by the drastic performance increase of all three AVES models when evaluated for the frog dataset.

For the self-supervised learning feature extractors, fine-tuning seems to only marginally improve performance.
The three feature extractors based on the AVES models all share the same general audio pretraining and architecture but differ largely in fine-tuning. 
The similarity in performance indicates that the dominant influence on the structuring of embeddings is defined by either the pretraining or the architecture.
Animal2vec\_XC and Animal2vec\_MK, the latter of which is fine-tuned, largely share the same architecture but were trained on very different datasets.
Yet, both models reach similarly poor performance, this is especially surprising for the fine-tuned Animal2vec\_MK.

%For the supervised learning feature extractor SurfPerch, which was developed for marine data in coral reefs, bird training data from Perch was mixed with coral reef sounds during pretraining.
%While the target domain is very different from the bird dataset, SurfPerch still reaches very high performance.
%This performance drops in terms of classification though, as soon as the target domain is shifted to the frog dataset.
%Again, this indicates that including data of the target domain is more effective during pretraining than fine-tuning.


While the dimensions of the feature extractors vary greatly, performance does not correlate with dimensionality.
Nonetheless, in Table \ref{tab:results} we demonstrated that using a standardized embedding space affects clustering and classification performance differently.
% The performance differences that can be observed are predominantly in clustering, indicating that the graph structure, which K-Means builds for the clustering, is aided by dimensionality reduction using UMAP.
% For this study dimensionality reduction using Principal Component Analysis was also performed and evaluated using linear classification, however, performance only changed marginally and was therefore omitted from this comparison.


% This analysis underlines the high quality of embeddings created by large supervised learning feature extractors like BirdNET and Perch.
While the bird data was in-domain for BirdNET and Perch, the same applies for Biolingual, Animal2vec\_XC and BirdAVES\_ESpecies, none of which reached similar performance metrics.
The decrease in classification performance by Perch on the frog data raises the question if this difference can be attributed to its training dataset.
Perch's training data, which is comprised of solely xeno-canto recordings perhaps makes it less domain agnostic than BirdNET which was trained on selected curated bird datasets along with a large portion of xeno-canto.


This study presents a workflow for in-depth analysis of embedding spaces, which can be reproduced with the provided repository bacpipe.
Evaluation through clustering and classification has shown to vary significantly when applied to the different evaluation sets, undermining the use of both metrics to better understand how training setup affects performance.
While the results presented here allow for further discussion, we believe streamlining the process of evaluating and comparing such a wide variety of feature extractors can help to investigate performance differences as a function of training paradigm and training domain.
By establishing a default analysis of feature extractors alongside the common classification benchmarks, bioacoustic research can accelerate towards a better understanding of what training characteristics are beneficial in this domain.

\section{Conclusion}
\label{sec:conslusion}

In this study we have compared a variety of different state-of-the-art bioacoustic deep learning models, representing different training paradigms and training domains.
To compare the models, we have isolated their feature extractors and used them to generate embeddings of two curated evaluation datasets consisting of annotated bird and frog sounds.
The aim of this study was to firstly present a large comparison of very different bioacoustic deep learning feature extractors and to evaluate how training setup affects performance.
Performance was evaluated through clustering using an AMI score and through kNN classification using a macro accuracy score.

We have shown that bioacoustic feature extractors still struggle with polyphonic PAM datasets, especially if they are outside of the training domain.
% At this point, self-supervised learning performance is still inferior to that of supervised learning models.
However, when comparing training paradigms,  supervised learning remains a strong method to pretrain a feature extractor for general use, despite advances in self-supervised learning.
This performance difference is visible in both kNN classification and even more in clustering.
Furthermore, we have shown that alignment of training domain and target domain during pretraining impacts performance more than during fine-tuning.
This study presents a roadmap for a more in-depth performance evaluation of bioacoustic deep learning models, allowing for a better understanding of how training setup impacts downstream performance.



\section{Acknowledgments}
A lot of work went into developing each of the feature extractors presented here, and we highly appreciate that they are made publicly available.
We would also like to acknowledge Álvaro Vega-Hidalgo and his coauthors, who collected, assembled and annotated the dataset used to evaluate the feature extractors. 
This project is funded by the Marie Skłodowska-Curie doctoral network BioAcousticAI.
% For bibtex users:
% \bibliography{FA2025_template}
\bibliography{References}


\end{document}
